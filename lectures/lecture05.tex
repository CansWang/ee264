\documentclass[10pt]{beamer}
\usefonttheme{professionalfonts}
%\usetheme{CambridgeUS}
%
% Choose how your presentation looks.
%
% For more themes, color themes and font themes, see:
% http://deic.uab.es/~iblanes/beamer_gallery/index_by_theme.html
%
\mode<presentation>
{
  \usetheme{default}      % or try Darmstadt, Madrid, Warsaw, ...
  \usecolortheme{beaver} % or try albatross, beaver, crane, ...
  \usefonttheme{default}  % or try serif, structurebold, ...
  \setbeamertemplate{navigation symbols}{}
  \setbeamertemplate{caption}[numbered]
} 

\usepackage[english]{babel}
\usepackage[utf8x]{inputenc}
\usepackage{tikz}
\usepackage{pgfplots}
\usepackage{array}  % for table column M
\usepackage{makecell} % to break line within a cell
\usepackage{verbatim}
\usepackage{graphicx}
\usepackage{epstopdf}
\usepackage{amsfonts}
\usepackage{xcolor}
%\captionsetup{compatibility=false}
%\usepackage{dsfont}
\usepackage[absolute,overlay]{textpos}
\usetikzlibrary{calc}
\usetikzlibrary{pgfplots.fillbetween, backgrounds}
\usetikzlibrary{positioning}
\usetikzlibrary{arrows}
\usetikzlibrary{pgfplots.groupplots}
\usetikzlibrary{arrows.meta}
\usetikzlibrary{plotmarks}

\usepgfplotslibrary{groupplots}
\pgfplotsset{compat=newest} 
%\pgfplotsset{plot coordinates/math parser=false}

\usepackage{hyperref}
\hypersetup{
    colorlinks=true,
    linkcolor=blue,
    filecolor=magenta,      
    urlcolor=cyan,
}

\definecolor{matlabcomment}{RGB}{34,139,34}

\pgfmathdeclarefunction{gauss}{2}{%
	\pgfmathparse{1/(#2*sqrt(2*pi))*exp(-((x-#1)^2)/(2*#2^2))}%
}

\pgfmathdeclarefunction{laplacian}{2}{%
	\pgfmathparse{1/(#2*2)*exp(-(abs(x-#1))/(#2))}%
}

\pgfmathdeclarefunction{pretty_func}{1}{%
	\pgfmathparse{cos(deg(#1/2)) - sin(deg(#1)) + cos(deg(#1/2)-45) - sin(deg(#1/4)-154)}%
}

\pgfplotsset{
	dirac/.style={
		mark=triangle*,
		mark options={scale=2},
		ycomb,
		scatter,
		visualization depends on={y/abs(y)-1 \as \sign},
		scatter/@pre marker code/.code={\scope[rotate=90*\sign,yshift=-2pt]}
	}
}

\newcommand\PlotSampledSpectrum[4]{%
	\def\fs{#2}%
	\def\fmax{#3}%
	\def\ros{#4}%
	\input{#1}%
}

\pgfmathdeclarefunction{invgauss}{2}{%
	\pgfmathparse{sqrt(-2*ln(#1))*cos(deg(2*pi*#2))}%
}

\tikzset{
	declare function={
		sinc(\x) = (and(\x!=0, 1) * (sin(deg(pi*\x))/(pi*\x)) +
		(and(\x==0, 1) * 1);
	}
}

\DeclareMathOperator{\E}{\mathbb{E}} % expectation

\newcolumntype{M}[1]{>{\centering\arraybackslash}m{#1}}

\definecolor{blue2}{RGB}{51, 105, 232}  
\definecolor{red2}{RGB}{213, 15, 37}  
\definecolor{green2}{RGB}{0, 153, 37}  
\definecolor{green3}{rgb}{0.1922, 0.6392, 0.3294}% 
\definecolor{yellow2}{RGB}{238, 178, 17} 
\definecolor{gray2}{RGB}{102, 102, 102}
\definecolor{orange2}{RGB}{230, 85, 13}

% Qualitative pallete set1 from www.ColorBrewer.org
\definecolor{Qred}{RGB}{228,26,28}
\definecolor{Qblue}{RGB}{55,126,184}
\definecolor{Qgreen}{RGB}{77,175,74}
\definecolor{Qpurple}{RGB}{152,78,163}
\definecolor{Qorange}{RGB}{255,127,0}
\definecolor{Qyellow}{RGB}{255,255,51}
\definecolor{Qbrown}{RGB}{166,86,40}
\definecolor{Qpink}{RGB}{247,129,191}
\definecolor{Qgray}{RGB}{153,153,153}

\title[EE 264]{Quantization}
\author{Jose Krause Perin}
\institute{Stanford University}
\date{July 13, 2017}

\begin{document}

\begin{frame}
  \titlepage
\end{frame}

% Uncomment these lines for an automatically generated outline.
%\begin{frame}{Outline}
%  \tableofcontents
%\end{frame}

\begin{frame}{Announcements}
	\begin{itemize}
		\item Homework \#2 due today
		\item Homework \#3 will be released today and it is due next Thursday
	\end{itemize}
\end{frame}

%
\begin{frame}{Practice and theory}
\begin{block}{In practice}
	\begin{center}
		\resizebox{\linewidth}{!}{\def\layersep{1.5cm}
\def\outsep{0.7cm}
\def\dy{1.25}

\begin{tikzpicture}[->, >=stealth, shorten >= 0pt, draw=black!50, node distance=\layersep, font=\sffamily]
    \tikzstyle{node}=[circle,fill=black,minimum size=2pt,inner sep=0pt]
    \tikzstyle{block}=[draw=black,rectangle,fill=none,minimum size=1.5cm, inner sep=0pt]
    \tikzstyle{annot} = []

	\node[node] (xc) at (0, -\dy cm) {};
    \node[block] (ADC) at (1*\layersep, -\dy cm) {ADC};
    \node[block, text width = 2.5cm, align= center] (DSP) at (3*\layersep, -\dy cm) {Digital Signal Processor};
    \node[block] (DAC) at (5*\layersep, -\dy cm) {DAC};
	\coordinate (yc) at (6*\layersep, -\dy cm) {};
	
	\coordinate (mid1) at ($(ADC.east)!0.5!(DSP.west)$) {};
	\coordinate (mid2) at ($(DSP.east)!0.5!(DAC.west)$) {};
		
    \path (xc) edge (ADC);
    \path (ADC) edge (DSP);
    \path (DSP) edge (DAC);
    \path (DAC) edge (yc);
    
    \node[above = 0.5mm of mid1] {$x[n]$};
    \node[above = 0.5mm of mid2] {$y[n]$};
    \node[left = 0mm of xc, text width = 1cm, align=center] {$x_c(t)$};
    \node[right = 0mm of yc, text width = 1cm, align=center] {$y_c(t)$}; 
    

\end{tikzpicture}}
	\end{center}
\end{block}

\begin{block}{DSP theory}
	\begin{center}
		\resizebox{\linewidth}{!}{\def\layersep{2cm}
\def\outsep{0.7cm}
\def\dy{1.25}

\begin{tikzpicture}[->, >=stealth, shorten >= 0pt, draw=black!50, node distance=\layersep, font=\sffamily]
    \tikzstyle{node}=[circle,fill=black,minimum size=2pt,inner sep=0pt]
    \tikzstyle{block}=[draw=black,rectangle,fill=none,minimum size=1.5cm, inner sep=0pt]
    \tikzstyle{annot} = []

	\node[node] (xc) at (0, -\dy cm) {};
    \node[block] (ADC) at (1*\layersep, -\dy cm) {C-to-D};
    \node[block, text width = 2cm, align= center] (DSP) at (3*\layersep, -\dy cm) {LTI \\ System};
    \node[block] (DAC) at (5*\layersep, -\dy cm) {D-to-C};
	\coordinate (yc) at (6*\layersep, -\dy cm) {};
	
	\coordinate (mid1) at ($(ADC.east)!0.5!(DSP.west)$) {};
	\coordinate (mid2) at ($(DSP.east)!0.5!(DAC.west)$) {};
		
    \path (xc) edge (ADC);
    \path (ADC) edge (DSP);
    \path (DSP) edge (DAC);
    \path (DAC) edge (yc);
    
    \node[above = 0.5mm of mid1] {$x[n]$};
    \node[below = 0.5mm of mid1] {$X(e^{j\omega})$};
    \node[above = 0.5mm of mid2] {$y[n]$};
    \node[below = 0.5mm of mid2] {$Y(e^{j\omega})$};
    \node[above = 0mm of xc, text width = 1cm, align=center] {$x_c(t)$};
    \node[below = 0mm of xc, text width = 1cm, align=center] {$X_c(j\Omega)$};
    \node[above = 0mm of yc, text width = 1cm, align=center] {$y_r(t)$}; 
    \node[below = 0mm of yc, text width = 1cm, align=center] {$Y_r(j\Omega)$};
    \node at ($(DSP.south)-(0, 0.25cm)$) {$h[n] \leftrightarrow H(e^{j\omega})$};
\end{tikzpicture}}
	\end{center}
	\textbf{Problem:} This simplified model does not account for \textbf{quantization} (today's lecture) or \textbf{finite precision arithmetic} (lecture 8)
\end{block}
\end{frame}

%
\begin{frame}{Including quantization}
	\begin{block}{Analog-to-digital converter}
		A more realistic model
		\begin{center}
		\resizebox{0.9\linewidth}{!}{
			\begin{tikzpicture}[->, >=stealth, shorten >= 0pt, draw=black!50, node distance=2.75cm, font=\sffamily]
				\tikzstyle{node}=[circle,fill=black,minimum size=2pt,inner sep=0pt]
				\tikzstyle{block}=[draw=black,rectangle,fill=none,minimum size=1.5cm, inner sep=0pt]
				
				\node[node] (xc) {};
				\node[block, right=1cm of xc] (CTD) {C-to-D};
				\node[block, right of=CTD, text width = 2cm, align= center] (Q) {Quantizer};
				\node[block, right of=Q] (coder) {Coder};
				\coordinate[right=1.5cm of coder] (yc) {};
				
				\coordinate (mid1) at ($(CTD.east)!0.5!(Q.west)$) {};
				\coordinate (mid2) at ($(Q.east)!0.5!(coder.west)$) {};
				
				\path (xc) edge (CTD);
				\path (CTD) edge (Q);
				\path (Q) edge (coder);
				\path (coder) edge (yc);
				
				\node[above = 0.5mm of mid1] {$x[n]$};
				\node[above = 0.5mm of mid2] {$x_Q[n]$};
				\node[above = 0mm of xc, text width = 1cm, align=center] {$x_c(t)$};
				\node[above = 0mm of yc, align=center] {$x_B[n]$}; 
				\node[below = 0mm of yc, text width = 2.5cm, align=center] {Binary \\ representation}; 
				
				\node[align=center] at ($(CTD.south)-(0, 0.4cm)$) {Sampling \\ $T$};
				\node[align=center, text width=4cm] at ($(Q.south)-(0, 1cm)$) {Analog-to-digital converter};
				
				\draw[dashed] ($(CTD.south west)-(0.25, 0.75)$) rectangle ($(coder.north east)+(0.25, 0.3)$) {};
				
			\end{tikzpicture}
		}
		\end{center}
	\end{block}
\end{frame}

%
\begin{frame}{Quantizer}
\begin{center}
	\resizebox{0.5\linewidth}{!}{
		\begin{tikzpicture}[->, >=stealth, shorten >= 0pt, draw=black!50, node distance=0.25cm, font=\sffamily]
		\tikzstyle{node}=[circle,fill=black,minimum size=2pt,inner sep=0pt]
		\tikzstyle{block}=[draw=black,rectangle,fill=none,minimum size=1cm, inner sep=0pt]
		\tikzstyle{annot} = []
		
		\node[node] (xc) {};
		\node[block, right=1cm of xc, text width = 1cm, align= center] (Q) {Q};
		\coordinate[right=1cm of Q] (yc) {};
			
		\path (xc) edge (Q);
		\path (Q) edge (yc);
		
		\node[above = 0mm of xc, text width = 1cm, align=center] {$x[n]$};
		\node[above = 0mm of yc, align=center] {$x_Q[n]$}; 	
		\end{tikzpicture}
	}
\end{center}

\begin{center}
	\resizebox{0.5\linewidth}{!}{
		\begin{tikzpicture}
		\begin{axis}[
			axis lines*=middle,
			enlargelimits = false, clip=true,
			ymin=-7, ymax=7,
			xmin=-7, xmax=7,
			axis line style={->,>=stealth},
			xlabel={$x[n]$},
			ylabel={$x_Q[n]$},
			yticklabel style = {yshift=0.2cm},
			xticklabel style = {yshift=-0.1cm},
			every axis x label/.style={
				at={(ticklabel* cs:1)},
				anchor=west,
			},
			every axis y label/.style={
				at={(ticklabel* cs:1)},
				anchor=south,
			},
			ytick={1},
			xtick=\empty,
			every outer y axis line/.append style={white!15!black},
			every y tick label/.append style={font=\color{white!15!black}},
			legend style={draw=white!15!black,fill=white,legend cell align=left}]			
			\addplot[line width=1pt] coordinates {(0, 0) (1, 0) (1, 1) (2, 1) (2, 2) (3, 2) (3, 3) (4, 3) (4, 4) (5, 4) (5, 5)};
			\addplot[black!20, line width=1pt, domain=-7:7, samples=2] {x};
		\end{axis}
		\end{tikzpicture}
	}
\end{center}

Terminology
\begin{itemize}
	\item $\Delta X$ is the \textbf{dynamic range}
	\item $q$ is the quantization resolution
\end{itemize}

The dynamic range, quantization resolution, and number of bits are related by 
\begin{equation*}
q = \frac{\Delta X}{2^{B}}
\end{equation*}

\end{frame}





%
\begin{frame}{Outline}
	\begin{itemize}
		\item Probabilistic interpretation of quantization
		\item Quantization noise shaping
	\end{itemize}
\end{frame}


%
\begin{frame}{Summary}
\begin{itemize}
	\item a
\end{itemize}
\end{frame}

\end{document}
