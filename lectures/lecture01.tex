\documentclass{beamer}
\usefonttheme{professionalfonts}
%\usetheme{CambridgeUS}
%
% Choose how your presentation looks.
%
% For more themes, color themes and font themes, see:
% http://deic.uab.es/~iblanes/beamer_gallery/index_by_theme.html
%
\mode<presentation>
{
  \usetheme{default}      % or try Darmstadt, Madrid, Warsaw, ...
  \usecolortheme{beaver} % or try albatross, beaver, crane, ...
  \usefonttheme{default}  % or try serif, structurebold, ...
  \setbeamertemplate{navigation symbols}{}
  \setbeamertemplate{caption}[numbered]
} 

\usepackage[english]{babel}
\usepackage[utf8x]{inputenc}
\usepackage{tikz}
\usepackage{pgfplots}
\usepackage{verbatim}
\usepackage{graphicx}
\usetikzlibrary{calc}
\usepgfplotslibrary{fillbetween}
\usetikzlibrary{positioning}

\usetikzlibrary{pgfplots.groupplots}
\usetikzlibrary{plotmarks}
\usetikzlibrary{calc}

\usepgfplotslibrary{groupplots}
\pgfplotsset{compat=newest} 
\pgfplotsset{plot coordinates/math parser=false}

\usepackage{hyperref}
\hypersetup{
    colorlinks=true,
    linkcolor=blue,
    filecolor=magenta,      
    urlcolor=cyan,
}

\title[EE 264]{EE 264: Digital Signal Processing}
\author{Jose Krause Perin}
\institute{Stanford University}
\date{\today}

\begin{document}

\begin{frame}
  \titlepage
\end{frame}

% Uncomment these lines for an automatically generated outline.
%\begin{frame}{Outline}
%  \tableofcontents
%\end{frame}

\section{Introduction}


\begin{frame}
\frametitle{Teaching staff}

\begin{block}{Instructor}
Jose Krause Perin \\
Office hours: Thursdays 2--4pm, \href{https://campus-map.stanford.edu/?id=04-040&lat=37.42879024&lng=-122.1740029&zoom=17&srch=spilker}{Spilker building}, room 216, or by appointment \\
e-mail: jkperin@stanford.edu
\end{block}
\vskip 1cm
\begin{block}{TA}
XX \\
Office hours: XX XXpm, Spilker building, room 216, or by appointment \\
e-mail: XX@stanford.edu
\end{block}
    	
\end{frame}

%
\begin{frame}{Why {\bf Digital} Signal Processing?}

\begin{itemize}
  \item Flexibility
  \item System/implementation does not age
  \item “Easy” implementation
  \item Reusable hardware
  \item Sophisticated processing
  \item Process on a computer
  \item (Today) Computation is cheaper and better
\end{itemize}

\end{frame}

%
\begin{frame}{Why learn DSP?}

\begin{itemize}
	\item Swiss-Army-Knife of modern EE
    \item Impacts all aspects of modern life
    \item Communications (wireless, internet, GPS...)
    \item Control and monitoring (cars, machines...)
    \item Multimedia (mp3, cameras, videos, restoration ...)
    \item Health (medical devices, imaging....)
    \item Economy (stock market, prediction)
    \item More....
\end{itemize}

\end{frame}

%
\begin{frame}{Example: speech recognition}

Siri

Spectrogram 

Recurrent neural net

\end{frame}

%
\begin{frame}{Example: image processing}

Cellphone camera prep-processing

\end{frame}

%
\begin{frame}{Example: digital communication}

OFDM, equalization

\end{frame}

%
\begin{frame}{What you'll learn}

\begin{itemize}
\item 
\end{itemize}

\end{frame}

%%%%%%%%%%%%%%%%%%%%%%%%%
\section{Administrative}

%
\begin{frame}{Administrative: resources}
\begin{block}{Required textbook}
\begin{itemize}
\item ``Discrete-time signal processing'', Oppenheim and Schafer, 3rd edition, 2010.
\item Available on 4-hour reserve at the \href{https://campus-map.stanford.edu/?id=04-080&lat=37.42787956&lng=-122.17429865&zoom=17&srch=engineeri}{Engineering Library (Terman)}
\end{itemize}
\end{block}

\begin{block}{Canvas: \href{https://canvas.stanford.edu/}{canvas.stanford.edu/}}
\begin{itemize}
\item Lecture notes, homework assignments, discussion
\item Submit homework on Canvas
\end{itemize}
\end{block}

\end{frame}

%
\begin{frame}{Administrative: tentative schedule}

\centering
\resizebox{\linewidth}{!}{
\begin{tabular}{c|c|l|c|c}
\hline
Date & Lecture	& Lecture Topic	& Reading\footnotemark & Assigned \\
\hline
27-Jun &	1	& Introduction and review	& Chaps. 2, 3	& \\
29-Jun	& 2	& Discrete-time random signals	& 2.10, App. A & HW1 \\
\hline
4-Jul	&  &	Independence Day, no classes & & \\
6-Jul	& 3 &	Sampling, reconstruction and DT filtering &	4.1--4.4 & HW2 \\
\hline
11-Jul	& 4 &	Changing the sampling rate by digital filtering	& 4.6, 4.7	& \\
13-Jul	& 5	& Quantization of samples, oversampling & 4.8, 4.9 &	HW3 \\
\hline
18-Jul	& 6	& Properties of LTI systems	& 5.1--5.7 & \\
20-Jul	& 7	& Linear Phase Systems, Digital Filter Structures &	6.1--6.6	& HW4 \\
\hline
25-Jul	& 8	& Quantization in digital filter implementations & 6.7--6.10	& \\
27-Jul	& 9	& Filter Design & Chap. 7	& HW5 \\
\hline
1-Aug	& &	In-class Midterm Exam & & \\
3-Aug	& 10 & The DFT and its properties, FFT	& 8.1--8.7, 9.1--9.3	& HW6 \\
\hline
8-Aug	& 11 & FFT and Spectrum analysis using the DFT	& 9.1--9.3, 10.1, 10.2 & \\
10-Aug	& 12 & Time-dependent spectrum analysis & 10.3--10.5 & HW7 \\
\hline
15-Aug	& 13 & Spectrum analysis of random signals	& 10.6--10.7	& \\
17-Aug & 14	& Parametric signal modeling	& 11.1--11.7 & 	\\
\hline
19-Aug	& & Final Exam	& \\
\hline
\end{tabular}
}
\footnotetext[1]{Reading assignments in ``Discrete-time signal processing'', Oppenheim and Schafer, 3rd edition}

\end{frame}

%
\begin{frame}{Administrative: grading basis}

\begin{block}{Homework assignments, $30\%$}
\begin{itemize} 
\item Homework will typically be released on Thursday and due on the following Thursday at 11:59pm.
\item Submissions should be made online on Canvas. Submit a single .pdf file with your solutions. 
\end{itemize}
\end{block}

\begin{block}{Midterm, $30\%$}
\begin{itemize} 
\item In-class midterm on August, 1st. 
\item Midterm covers lectures 1 to 9 (homework 1 to 4)
\item Midterm is open book and open notes
\end{itemize}
\end{block}

\begin{block}{Final, $40\%$}
\begin{itemize} 
\item Final exam on XXX
\item Final exam is open book and open notes
\end{itemize}
\end{block}

\end{frame}

%%%%%%%%%%%%%%%%%%%%%%%%%
\section{Review}

\begin{frame}{Review}

\begin{itemize}
\item Sampling and reconstruction
\item Discrete-time signals and systems
\item Signal representations: impulse response, frequency response, difference equation
\item The $z$-transform
\item Discrete-time Fourier transform
\item Discrete Fourier transform and FFT
\end{itemize}

\end{frame}

\subsection{Sampling and reconstruction}
%
\begin{frame}{Digital processing of analog signals}

\begin{figure}[t!]
	\centering
	\resizebox{\linewidth}{!}{\def\layersep{1.5cm}
\def\outsep{0.7cm}
\def\dy{1.25}

\begin{tikzpicture}[->, >=stealth, shorten >= 0pt, draw=black!50, node distance=\layersep, font=\sffamily]
    \tikzstyle{node}=[circle,fill=black,minimum size=2pt,inner sep=0pt]
    \tikzstyle{block}=[draw=black,rectangle,fill=none,minimum size=1.5cm, inner sep=0pt]
    \tikzstyle{annot} = []

	\node[node] (xc) at (0, -\dy cm) {};
    \node[block] (ADC) at (1*\layersep, -\dy cm) {ADC};
    \node[block, text width = 2.5cm, align= center] (DSP) at (3*\layersep, -\dy cm) {Digital Signal Processor};
    \node[block] (DAC) at (5*\layersep, -\dy cm) {DAC};
	\coordinate (yc) at (6*\layersep, -\dy cm) {};
	
	\coordinate (mid1) at ($(ADC.east)!0.5!(DSP.west)$) {};
	\coordinate (mid2) at ($(DSP.east)!0.5!(DAC.west)$) {};
		
    \path (xc) edge (ADC);
    \path (ADC) edge (DSP);
    \path (DSP) edge (DAC);
    \path (DAC) edge (yc);
    
    \node[above = 0.5mm of mid1] {$x[n]$};
    \node[above = 0.5mm of mid2] {$y[n]$};
    \node[left = 0mm of xc, text width = 1cm, align=center] {$x_c(t)$};
    \node[right = 0mm of yc, text width = 1cm, align=center] {$y_c(t)$}; 
    

\end{tikzpicture}}
	\label{fig:adc-dsp-dac}
\end{figure}

\begin{block}{A-to-D: analog-to-digital converter}
Filtering, sampling, and quantization
\end{block}

\begin{block}{Digital signal processor}
\begin{itemize} \itemsep 0pt
	\item Performs some operation e.g., convolution, DFT
	\item May be implemented on PCs with 64-bit floating-point precision, or may be implemented on DSP chips or ASICs with finite arithmetic precision (e.g., 6 bits of resolution).
\end{itemize}
\end{block}
\begin{block}{D-to-A: digital-to-analog converter}
Quantization and filtering
\end{block}

\end{frame}

%
\begin{frame}{Sampling}




\end{frame}

%
\begin{frame}{Discrete-time Fourier transform (DTFT)}

\begin{block}{Definition}
\begin{equation} \tag{Direct transform}
X(e^{j\omega}) = \sum_{n=-\infty}^{\infty} x[n]e^{-j\omega n} 
\end{equation}

\begin{equation}\tag{Inverse transform}
x[n] = \frac{1}{2\pi}\int_{-\pi}^{\pi}X(e^{j\omega})e^{j\omega n}\mathrm{d}\omega
\end{equation}
\end{block}

\begin{block}{Properties}
The DTFT is a periodic function
\begin{equation*}
X(e^{j(\omega + 2\pi)}) = X(e^{j\omega})
\end{equation*}
\end{block}
\end{frame}

%
\begin{frame}{Example: the ideal lowpass filter}

\begin{figure}
	\centering
	\resizebox{0.5\linewidth}{!}{\begin{tikzpicture} 
\only<1|handout:1>{
\begin{axis}[
name=ideal_lpf_td,
anchor=origin,
axis lines*=middle,
enlargelimits = true,
ymin=-0.1,
ymax=1.2*0.5,
xmin=-10,
xmax=10,
axis line style={->,>=stealth},
xlabel={$n$},
ylabel={$h_{lpf}[n]$},
yticklabel style = {yshift=0.2cm},
xticklabel style = {yshift=-0.1cm},
every axis x label/.style={
	at={(ticklabel* cs:1)},
	anchor=north,
},
every axis y label/.style={
	at={(ticklabel* cs:1)},
	anchor=south,
},
ytick=0.5,
yticklabels={$\frac{\omega_c}{\pi}$},
xtick=\empty,
every outer y axis line/.append style={white!15!black},
every y tick label/.append style={font=\color{white!15!black}},
legend style={draw=white!15!black,fill=white,legend cell align=left}]

\addplot[ycomb, mark=*, fill=white, mark options={scale=1.5, fill=white}, line width=1.5pt, domain=-10:10, samples=21] {sin(deg(pi/2*x))/(pi*x) + 1/2*(x == 0)};
\addplot[smooth, black!20, line width=1pt, domain=-10:10, samples=21] {sin(deg(pi/2*x))/(pi*x) + 1/2*(x == 0)};
\end{axis}

\begin{axis}[
name=ideal_lpf_fd,
at=(ideal_lpf_td.right of south east), anchor=left of south west,
axis lines*=middle,
enlargelimits = true,
xmax=4,
xmin=-4,
ymin=-0.2,
ymax=1.2,
axis line style={->,>=stealth},
xlabel={$\omega$},
ylabel={$H_{lpf}(e^{j\omega})$},
yticklabel style = {yshift=0.2cm},
xticklabel style = {yshift=-0.3cm},
every axis x label/.style={
    at={(ticklabel* cs:1)},
    anchor=north,
},
every axis y label/.style={
    at={(ticklabel* cs:1)},
    anchor=south,
},
xtick=\empty,
ytick={1},
xtick={-3.14, -1.5708, 1.5708, 3.14},
xticklabels={$-\pi$, $-\omega_c$, $\omega_c$, $\pi$},
every outer y axis line/.append style={white!15!black},
every y tick label/.append style={font=\color{white!15!black}},
legend style={draw=white!15!black,fill=white,legend cell align=left}]

	\addplot[black, domain=-pi/2:pi/2, samples=2,line width=2pt] {1};
	\addplot[black, line width=2pt] coordinates {(-pi/2, 0) (-pi/2, 1.01)};
	\addplot[black, domain=-pi:-pi/2, samples=2,line width=2pt] {0};
	\addplot[black, domain=pi/2:pi, samples=2,line width=2pt] {0};
	\addplot[black, line width=2pt] coordinates {(pi/2, 0) (pi/2, 1.01)};
\end{axis}
}

%%%%%%%%%%%%%%%%%% 
\only<2|handout:2>{
	\begin{axis}[
	name=truncated_lpf_td,
	anchor=origin,
	axis lines*=middle,
	enlargelimits = true,
	ymin=-0.1,
	ymax=1.2*0.5,
	xmin=-7,
	xmax=7,
	axis line style={->,>=stealth},
	xlabel={$n$},
	ylabel={$h_{lpf}[n]$},
	yticklabel style = {yshift=0.2cm},
	xticklabel style = {yshift=0.6cm},
	every axis x label/.style={
		at={(ticklabel* cs:1)},
		anchor=north,
	},
	every axis y label/.style={
		at={(ticklabel* cs:1)},
		anchor=south,
	},
	ytick=0.5,
	yticklabels={$\frac{\omega_c}{\pi}$},
	xtick={-7, 7},
	xticklabels={\tikz[baseline]{\node[fill=blue!20,anchor=base,scale=0.7] {$-M$};}, \tikz[baseline]{\node[fill=blue!20,anchor=base,scale=0.7] {$M$};}},
	every outer y axis line/.append style={white!15!black},
	every y tick label/.append style={font=\color{white!15!black}},
	legend style={draw=white!15!black,fill=white,legend cell align=left}]
	
	\addplot[ycomb, mark=*, fill=white, mark options={scale=1.5, fill=white}, line width=1.5pt, domain=-7:7, samples=15] {sin(deg(pi/2*x))/(pi*x) + 1/2*(x == 0)};
	\addplot[smooth, black!20, line width=1pt, domain=-7:7, samples=15] {sin(deg(pi/2*x))/(pi*x) + 1/2*(x == 0)};
	\end{axis}
	
	\begin{axis}[
	name=truncated_lpf_fd,
	at=(truncated_lpf_td.right of south east), anchor=left of south west,
	axis lines*=middle,
	enlargelimits = true,
	xmax=4,
	xmin=-4,
	ymin=-0.2,
	ymax=1.2,
	axis line style={->,>=stealth},
	xlabel={$\omega$},
	ylabel={$H_{lpf}(e^{j\omega})$},
	yticklabel style = {yshift=0.2cm},
	xticklabel style = {yshift=-0.3cm},
	every axis x label/.style={
		at={(ticklabel* cs:1)},
		anchor=north,
	},
	every axis y label/.style={
		at={(ticklabel* cs:1)},
		anchor=south,
	},
	xtick=\empty,
	ytick={1},
	xtick={-3.14, -1.5708, 1.5708, 3.14},
	xticklabels={$-\pi$, $-\omega_c$, $\omega_c$, $\pi$},
	every outer y axis line/.append style={white!15!black},
	every y tick label/.append style={font=\color{white!15!black}},
	legend style={draw=white!15!black,fill=white,legend cell align=left}]
	
	\addplot[black!20, domain=-pi/2:pi/2, samples=2,line width=2pt] {1};
	\addplot[black!20, line width=2pt] coordinates {(-pi/2, 0) (-pi/2, 1.01)};
	\addplot[black!20, domain=-pi:-pi/2, samples=2,line width=2pt] {0};
	\addplot[black!20, domain=pi/2:pi, samples=2,line width=2pt] {0};
	\addplot[black!20, line width=2pt] coordinates {(pi/2, 0) (pi/2, 1.01)};
	
	\addplot [color=black, solid, line width=1.5pt, forget plot]
	table[row sep=crcr]{
		-3.1416 0.039209 \\
		-3.0781 0.034184 \\
		-3.0147 0.020324 \\
		-2.9512 0.00099044 \\
		-2.8877 -0.019066 \\
		-2.8243 -0.0348 \\
		-2.7608 -0.042056 \\
		-2.6973 -0.038583 \\
		-2.6339 -0.024655 \\
		-2.5704 -0.0031491 \\
		-2.5069 0.020968 \\
		-2.4435 0.041656 \\
		-2.38 0.053128 \\
		-2.3165 0.051242 \\
		-2.2531 0.034652 \\
		-2.1896 0.0054863 \\
		-2.1261 -0.030621 \\
		-2.0627 -0.065184 \\
		-1.9992 -0.088127 \\
		-1.9357 -0.089437 \\
		-1.8723 -0.060929 \\
		-1.8088 0.0022081 \\
		-1.7453 0.10035 \\
		-1.6819 0.22911 \\
		-1.6184 0.37975 \\
		-1.5549 0.54037 \\
		-1.4915 0.69762 \\
		-1.428 0.83864 \\
		-1.3645 0.95293 \\
		-1.3011 1.0338 \\
		-1.2376 1.0793 \\
		-1.1741 1.0921 \\
		-1.1107 1.0788 \\
		-1.0472 1.0487 \\
		-0.98373 1.0122 \\
		-0.92026 0.97863 \\
		-0.8568 0.95526 \\
		-0.79333 0.94598 \\
		-0.72986 0.95114 \\
		-0.6664 0.96788 \\
		-0.60293 0.99105 \\
		-0.53947 1.0146 \\
		-0.476 1.0328 \\
		-0.41253 1.0417 \\
		-0.34907 1.0397 \\
		-0.2856 1.0278 \\
		-0.22213 1.0093 \\
		-0.15867 0.98893 \\
		-0.0952 0.97181 \\
		-0.031733 0.96207 \\
		0.031733 0.96207 \\
		0.0952 0.97181 \\
		0.15867 0.98893 \\
		0.22213 1.0093 \\
		0.2856 1.0278 \\
		0.34907 1.0397 \\
		0.41253 1.0417 \\
		0.476 1.0328 \\
		0.53947 1.0146 \\
		0.60293 0.99105 \\
		0.6664 0.96788 \\
		0.72986 0.95114 \\
		0.79333 0.94598 \\
		0.8568 0.95526 \\
		0.92026 0.97863 \\
		0.98373 1.0122 \\
		1.0472 1.0487 \\
		1.1107 1.0788 \\
		1.1741 1.0921 \\
		1.2376 1.0793 \\
		1.3011 1.0338 \\
		1.3645 0.95293 \\
		1.428 0.83864 \\
		1.4915 0.69762 \\
		1.5549 0.54037 \\
		1.6184 0.37975 \\
		1.6819 0.22911 \\
		1.7453 0.10035 \\
		1.8088 0.0022081 \\
		1.8723 -0.060929 \\
		1.9357 -0.089437 \\
		1.9992 -0.088127 \\
		2.0627 -0.065184 \\
		2.1261 -0.030621 \\
		2.1896 0.0054863 \\
		2.2531 0.034652 \\
		2.3165 0.051242 \\
		2.38 0.053128 \\
		2.4435 0.041656 \\
		2.5069 0.020968 \\
		2.5704 -0.0031491 \\
		2.6339 -0.024655 \\
		2.6973 -0.038583 \\
		2.7608 -0.042056 \\
		2.8243 -0.0348 \\
		2.8877 -0.019066 \\
		2.9512 0.00099044 \\
		3.0147 0.020324 \\
		3.0781 0.034184 \\
		3.1416 0.039209 \\
	};
	
	\node at (axis cs: 3.14, 1.1) {\Large \tikz[baseline]{
			\node[fill=blue!20,anchor=base] (t1) {$M = 7$}}};
	
	\end{axis}
}

%%%%%%%%%%%%%%%%%% 
\only<3|handout:3>{
	\begin{axis}[
	name=truncated_lpf_td,
	anchor=origin,
	axis lines*=middle,
	enlargelimits = true,
	ymin=-0.1,
	ymax=1.2*0.5,
	xmin=-19,
	xmax=19,
	axis line style={->,>=stealth},
	xlabel={$n$},
	ylabel={$h_{lpf}[n]$},
	yticklabel style = {yshift=0.2cm},
	xticklabel style = {yshift=0.6cm},
	every axis x label/.style={
		at={(ticklabel* cs:1)},
		anchor=north,
	},
	every axis y label/.style={
		at={(ticklabel* cs:1)},
		anchor=south,
	},
	ytick=0.5,
	yticklabels={$\frac{\omega_c}{\pi}$},
	xtick={-19, 19},
	xticklabels={\tikz[baseline]{\node[fill=blue!20,anchor=base,scale=0.7] {$-M$};}, \tikz[baseline]{\node[fill=blue!20,anchor=base,scale=0.7] {$M$};}},
	every outer y axis line/.append style={white!15!black},
	every y tick label/.append style={font=\color{white!15!black}},
	legend style={draw=white!15!black,fill=white,legend cell align=left}]
	
	\addplot[ycomb, mark=*, fill=white, mark options={scale=1.5, fill=white}, line width=1.5pt,  domain=-19:19, samples=39] {sin(deg(pi/2*x))/(pi*x) + 1/2*(x == 0)};
	\addplot[smooth, black!20, line width=1pt, domain=-19:19, samples=39] {sin(deg(pi/2*x))/(pi*x) + 1/2*(x == 0)};
	\end{axis}
	
	\begin{axis}[
	name=truncated_lpf_fd,
	at=(truncated_lpf_td.right of south east), anchor=left of south west,
	axis lines*=middle,
	enlargelimits = true,
	xmax=4,
	xmin=-4,
	ymin=-0.2,
	ymax=1.2,
	axis line style={->,>=stealth},
	xlabel={$\omega$},
	ylabel={$H_{lpf}(e^{j\omega})$},
	yticklabel style = {yshift=0.2cm},
	xticklabel style = {yshift=-0.3cm},
	every axis x label/.style={
		at={(ticklabel* cs:1)},
		anchor=north,
	},
	every axis y label/.style={
		at={(ticklabel* cs:1)},
		anchor=south,
	},
	xtick=\empty,
	ytick={1},
	xtick={-3.14, -1.5708, 1.5708, 3.14},
	xticklabels={$-\pi$, $-\omega_c$, $\omega_c$, $\pi$},
	every outer y axis line/.append style={white!15!black},
	every y tick label/.append style={font=\color{white!15!black}},
	legend style={draw=white!15!black,fill=white,legend cell align=left}]
	
	\addplot[black!20, domain=-pi/2:pi/2, samples=2,line width=2pt] {1};
	\addplot[black!20, line width=2pt] coordinates {(-pi/2, 0) (-pi/2, 1.01)};
	\addplot[black!20, domain=-pi:-pi/2, samples=2,line width=2pt] {0};
	\addplot[black!20, domain=pi/2:pi, samples=2,line width=2pt] {0};
	\addplot[black!20, line width=2pt] coordinates {(pi/2, 0) (pi/2, 1.01)};
	
	\addplot [color=black, solid, line width=1.5pt, forget plot]
table[row sep=crcr]{
	-3.1416 0.015876 \\
	-3.11 0.012806 \\
	-3.0784 0.0047726 \\
	-3.0469 -0.0051424 \\
	-3.0153 -0.013122 \\
	-2.9837 -0.01607 \\
	-2.9521 -0.012801 \\
	-2.9206 -0.0045107 \\
	-2.889 0.0056547 \\
	-2.8574 0.013781 \\
	-2.8259 0.016676 \\
	-2.7943 0.013113 \\
	-2.7627 0.0043342 \\
	-2.7311 -0.006365 \\
	-2.6996 -0.014863 \\
	-2.668 -0.017776 \\
	-2.6364 -0.013789 \\
	-2.6048 -0.0042266 \\
	-2.5733 0.0073679 \\
	-2.5417 0.016523 \\
	-2.5101 0.019535 \\
	-2.4785 0.014937 \\
	-2.447 0.0041704 \\
	-2.4154 -0.0088347 \\
	-2.3838 -0.019057 \\
	-2.3522 -0.022276 \\
	-2.3207 -0.016768 \\
	-2.2891 -0.004133 \\
	-2.2575 0.011106 \\
	-2.226 0.023056 \\
	-2.1944 0.026652 \\
	-2.1628 0.019706 \\
	-2.1312 0.0040161 \\
	-2.0997 -0.01496 \\
	-2.0681 -0.029875 \\
	-2.0365 -0.034159 \\
	-2.0049 -0.024686 \\
	-1.9734 -0.0034348 \\
	-1.9418 0.022586 \\
	-1.9102 0.043309 \\
	-1.8786 0.049014 \\
	-1.8471 0.034142 \\
	-1.8155 0.00025733 \\
	-1.7839 -0.042935 \\
	-1.7523 -0.079062 \\
	-1.7208 -0.08885 \\
	-1.6892 -0.055218 \\
	-1.6576 0.031713 \\
	-1.6261 0.1712 \\
	-1.5945 0.35111 \\
	-1.5629 0.55018 \\
	-1.5313 0.74274 \\
	-1.4998 0.90461 \\
	-1.4682 1.0186 \\
	-1.4366 1.0783 \\
	-1.405 1.0884 \\
	-1.3735 1.0631 \\
	-1.3419 1.0211 \\
	-1.3103 0.98084 \\
	-1.2787 0.95576 \\
	-1.2472 0.95147 \\
	-1.2156 0.96573 \\
	-1.184 0.99043 \\
	-1.1524 1.0152 \\
	-1.1209 1.0311 \\
	-1.0893 1.0336 \\
	-1.0577 1.0234 \\
	-1.0261 1.0055 \\
	-0.99457 0.98732 \\
	-0.963 0.97552 \\
	-0.93143 0.97388 \\
	-0.89985 0.98214 \\
	-0.86828 0.99648 \\
	-0.83671 1.0111 \\
	-0.80513 1.0206 \\
	-0.77356 1.0217 \\
	-0.74198 1.0146 \\
	-0.71041 1.0024 \\
	-0.67884 0.98987 \\
	-0.64726 0.98183 \\
	-0.61569 0.98105 \\
	-0.58412 0.98748 \\
	-0.55254 0.99841 \\
	-0.52097 1.0095 \\
	-0.48939 1.0166 \\
	-0.45782 1.0172 \\
	-0.42625 1.0111 \\
	-0.39467 1.001 \\
	-0.3631 0.99079 \\
	-0.33152 0.98432 \\
	-0.29995 0.98398 \\
	-0.26838 0.98979 \\
	-0.2368 0.99942 \\
	-0.20523 1.0091 \\
	-0.17366 1.0152 \\
	-0.14208 1.0154 \\
	-0.11051 1.0096 \\
	-0.078934 1.0002 \\
	-0.047361 0.99075 \\
	-0.015787 0.98491 \\
	0.015787 0.98491 \\
	0.047361 0.99075 \\
	0.078934 1.0002 \\
	0.11051 1.0096 \\
	0.14208 1.0154 \\
	0.17366 1.0152 \\
	0.20523 1.0091 \\
	0.2368 0.99942 \\
	0.26838 0.98979 \\
	0.29995 0.98398 \\
	0.33152 0.98432 \\
	0.3631 0.99079 \\
	0.39467 1.001 \\
	0.42625 1.0111 \\
	0.45782 1.0172 \\
	0.48939 1.0166 \\
	0.52097 1.0095 \\
	0.55254 0.99841 \\
	0.58412 0.98748 \\
	0.61569 0.98105 \\
	0.64726 0.98183 \\
	0.67884 0.98987 \\
	0.71041 1.0024 \\
	0.74198 1.0146 \\
	0.77356 1.0217 \\
	0.80513 1.0206 \\
	0.83671 1.0111 \\
	0.86828 0.99648 \\
	0.89985 0.98214 \\
	0.93143 0.97388 \\
	0.963 0.97552 \\
	0.99457 0.98732 \\
	1.0261 1.0055 \\
	1.0577 1.0234 \\
	1.0893 1.0336 \\
	1.1209 1.0311 \\
	1.1524 1.0152 \\
	1.184 0.99043 \\
	1.2156 0.96573 \\
	1.2472 0.95147 \\
	1.2787 0.95576 \\
	1.3103 0.98084 \\
	1.3419 1.0211 \\
	1.3735 1.0631 \\
	1.405 1.0884 \\
	1.4366 1.0783 \\
	1.4682 1.0186 \\
	1.4998 0.90461 \\
	1.5313 0.74274 \\
	1.5629 0.55018 \\
	1.5945 0.35111 \\
	1.6261 0.1712 \\
	1.6576 0.031713 \\
	1.6892 -0.055218 \\
	1.7208 -0.08885 \\
	1.7523 -0.079062 \\
	1.7839 -0.042935 \\
	1.8155 0.00025733 \\
	1.8471 0.034142 \\
	1.8786 0.049014 \\
	1.9102 0.043309 \\
	1.9418 0.022586 \\
	1.9734 -0.0034348 \\
	2.0049 -0.024686 \\
	2.0365 -0.034159 \\
	2.0681 -0.029875 \\
	2.0997 -0.01496 \\
	2.1312 0.0040161 \\
	2.1628 0.019706 \\
	2.1944 0.026652 \\
	2.226 0.023056 \\
	2.2575 0.011106 \\
	2.2891 -0.004133 \\
	2.3207 -0.016768 \\
	2.3522 -0.022276 \\
	2.3838 -0.019057 \\
	2.4154 -0.0088347 \\
	2.447 0.0041704 \\
	2.4785 0.014937 \\
	2.5101 0.019535 \\
	2.5417 0.016523 \\
	2.5733 0.0073679 \\
	2.6048 -0.0042266 \\
	2.6364 -0.013789 \\
	2.668 -0.017776 \\
	2.6996 -0.014863 \\
	2.7311 -0.006365 \\
	2.7627 0.0043342 \\
	2.7943 0.013113 \\
	2.8259 0.016676 \\
	2.8574 0.013781 \\
	2.889 0.0056547 \\
	2.9206 -0.0045107 \\
	2.9521 -0.012801 \\
	2.9837 -0.01607 \\
	3.0153 -0.013122 \\
	3.0469 -0.0051424 \\
	3.0784 0.0047726 \\
	3.11 0.012806 \\
	3.1416 0.015876 \\
};
	
	\node at (axis cs: 3.14, 1.1) {\Large \tikz[baseline]{
			\node[fill=blue!20,anchor=base] (t1) {$M = 19$}}};
	
	\end{axis}
}

\end{tikzpicture}}
	\label{fig:ideal-lpf}
\end{figure}


\end{frame}

%
\begin{frame}{DTFT properties}
\begin{itemize}
\item Linearity
\begin{equation*}
ax_1[n]  + bx_2[n] \Longleftrightarrow aX_1(e^{j\omega})+bX_2(e^{j\omega})
\end{equation*}
\item Time shift, delay for $n_d > 0$ delay, advance for $n_d < 0$
\begin{equation*}
x[n-n_d] \Longleftrightarrow e^{-j\omega n_d}X(e^{j\omega})
\end{equation*}
\item Frequency shift (modulation)
\begin{equation*}
e^{j\omega_0n}x[n] \Longleftrightarrow X(e^{j(\omega-\omega_0)})
\end{equation*}
\item Convolution
\begin{equation*}
y[n] = x[n]\ast h[n] \Longleftrightarrow Y(e^{j\omega}) = X(e^{j\omega})H(e^{j\omega})
\end{equation*}
\item Time reversal
\begin{equation*}
x[-n] \Longleftrightarrow X(e^{-j\omega})
\end{equation*}

\end{itemize}
\end{frame}

%
\begin{frame}{DTFT properties}
\begin{itemize}
	\item Linear weighting
	\begin{equation*}
	nx[n] \Longleftrightarrow j\frac{dX(e^{j\omega})}{d\omega}
	\end{equation*}
	\item Multiplication of sequences (windowing)
	\begin{equation*}
	x[n]w[n] \Longleftrightarrow \frac{1}{2\pi}\int_{-\pi}^{\pi}X(e^{j\theta})W(e^{j(\omega-\theta)}){d\theta}
	\end{equation*}
	\item Parseval's theorem
	\begin{equation*}
	\sum_{n=-\infty}^{\infty}x[n]y^*[n] = \frac{1}{2\pi}\int_{-\pi}^{\pi}X(e^{j\omega})Y^*(e^{j\omega})\mathrm{d}\omega
	\end{equation*}
	\item Signal energy
	\begin{equation*}
	\sum_{n=-\infty}^{\infty}|x[n]|^2 = \frac{1}{2\pi}\int_{-\pi}^{\pi}|X(e^{j\omega})|^2\mathrm{d}\omega
	\end{equation*}	
	\item Deterministic autocorrelation function
	\begin{equation*}
	c_{xx}[n] = \sum_{m=-\infty}^{\infty}x[m]x[n+m] \Longleftrightarrow X(e^{j\omega})X^*(e^{j\omega}) = |X(e^{j\omega})|^2
	\end{equation*}	
\end{itemize}
\end{frame}

\begin{frame}{Obtaining frequency response from difference equation}

\begin{block}{Difference equation}
\begin{equation*}
\sum_{k=0}^N a_k y[n-k] = \sum_{k=0}^Mb_k x[n-k]
\end{equation*}
\end{block}

Using the linearity and time shift properties of the DTFT it follows that

\begin{align*}
\sum_{k=0}^N a_kY(e^{j\omega})e^{-j\omega k} = \sum_{k=0}^M b_kX(e^{j\omega})e^{-j\omega k} \\
\frac{Y(e^{j\omega})}{X(e^{j\omega})} = \frac{\sum_{k=0}^M b_ke^{-j\omega k}}{\sum_{k=0}^N a_ke^{-j\omega k}}
\end{align*}

\end{frame}


\begin{frame}{The $z$-Transform}

\begin{block}{Definition}
	\begin{equation} \tag{Direct transform}
	X(z) = \sum_{n=-\infty}^{\infty} x[n]z^{-n} 
	\end{equation}
	
	\begin{equation}\tag{Inverse transform}
	x[n] = \frac{1}{2j\pi}\int_{C}X(z)z^{n-1}\mathrm{d}z
	\end{equation}
\end{block}

\begin{itemize}
	\item Note that the infinite sum in the direct transform may not converge for all values of $z$. We must always specify for which values of $z$ the sum exist, which is known as \textbf{region of converge (ROC)}.
	\item The inverse transform is given by the contour integral over some path $C$. This is generally laborious, so we'll obtain the inverse $z$-transform using indirect ways.
\end{itemize}
\end{frame}

\begin{frame}{Region of convergence}

Defined as the values of $z$ for which

\begin{equation*}
|X(z)| = \bigg|\sum_{n=-\infty}^{\infty}x[n]z^{-n}\bigg| \leq \sum_{n=-\infty}^{\infty} |x[n]||z|^{-n} < \infty
\end{equation*}

As long as $|z|^{-n}$ decays faster than $|x[n]|$ grows, the sum will be bounded.

\end{frame}


\begin{frame}{Region of convergence}

\begin{block}{Example 1: right-sided (causal) exponential signal}
Right-sided exponential signal: $x[n] = a^nu[n]$
\begin{equation} 
X(z) = \sum_{n=0}^\infty a^nz^{-n} = (az^{-1})^n 
\end{equation}

This sum converges only if $|az^{-1}| < 1$ geometric progression:
\begin{equation} 
X(z) = \frac{1}{1-az^{-1}} = \frac{z}{z-a}
\end{equation}

	
\end{block}

\end{frame}

\end{document}
