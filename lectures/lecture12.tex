\documentclass[10pt]{beamer}
\usefonttheme{professionalfonts}
%\usetheme{CambridgeUS}
%
% Choose how your presentation looks.
%
% For more themes, color themes and font themes, see:
% http://deic.uab.es/~iblanes/beamer_gallery/index_by_theme.html
%
\mode<presentation>
{
  \usetheme{default}      % or try Darmstadt, Madrid, Warsaw, ...
  \usecolortheme{beaver} % or try albatross, beaver, crane, ...
  \usefonttheme{default}  % or try serif, structurebold, ...
  \setbeamertemplate{navigation symbols}{}
  \setbeamertemplate{caption}[numbered]
} 

\usepackage[english]{babel}
\usepackage[utf8x]{inputenc}
\usepackage{tikz}
\usepackage{pgfplots}
\usepackage{array}  % for table column M
\usepackage{makecell} % to break line within a cell
\usepackage{verbatim}
\usepackage{graphicx}
\usepackage{epstopdf}
\usepackage{amsfonts}
\usepackage{xcolor}
\usepackage{ifthen}
%\usepackage{mathtools}
\usepackage[makeroom]{cancel}
\usetikzlibrary{spy}
%\captionsetup{compatibility=false}
%\usepackage{dsfont}
\usepackage[absolute,overlay]{textpos}
\usetikzlibrary{calc, angles,quotes}
\usetikzlibrary{pgfplots.fillbetween, backgrounds}
\usetikzlibrary{positioning}
\usetikzlibrary{arrows}
\usetikzlibrary{pgfplots.groupplots}
\usetikzlibrary{arrows.meta}
\usetikzlibrary{plotmarks}
\usetikzlibrary{decorations.markings}
\usepgfplotslibrary{groupplots}
\pgfplotsset{compat=newest} 
%\pgfplotsset{plot coordinates/math parser=false}

\usepackage{hyperref}
\hypersetup{
    colorlinks=true,
    linkcolor=blue,
    filecolor=magenta,      
    urlcolor=cyan,
}

%
%\def\EXTERNALIZE{1}
%%% Page numbering
\usepackage{etoolbox} % necessary for excluding beamer-only frames from page numbering

\makeatletter
\pretocmd{\beamer@@@@frame}{\alt<#1>{}{\beamer@noframenumberingtrue}}{}{}
\makeatother

\addtobeamertemplate{navigation symbols}{}{%
	\usebeamerfont{footline}%
	\usebeamercolor[fg]{footline}%
	\hspace{1em}%
	\insertframenumber/\inserttotalframenumber
}
%%%

\definecolor{matlabcomment}{RGB}{34,139,34}

\pgfmathdeclarefunction{gauss}{1}{%
	\pgfmathparse{1/(sqrt(2*pi))*exp(-((#1)^2)/2)}%
}

\pgfmathdeclarefunction{laplacian}{2}{%
	\pgfmathparse{1/(#2*2)*exp(-(abs(x-#1))/(#2))}%
}

\pgfmathdeclarefunction{pretty_func}{1}{%
	\pgfmathparse{cos(deg(#1/2)) - sin(deg(#1)) + cos(deg(#1/2)-45) - sin(deg(#1/4)-154)}%
}

\pgfplotsset{
	dirac/.style={
		mark=triangle*,
		mark options={scale=2},
		ycomb,
		scatter,
		visualization depends on={y/abs(y)-1 \as \sign},
		scatter/@pre marker code/.code={\scope[rotate=90*\sign,yshift=-2pt]}
	}
}

\def\thickness{very thick}

\tikzset{
amark/.style 2 args={
	decoration={             
		markings, 
		mark=at position {0.5} with { 
			\arrow{stealth},
			\node[#2] {#1};
		}
	}, \thickness,
	postaction={decorate}
},
earlymark/.style 2 args={
	decoration={             
		markings, 
		mark=at position {0.25} with { 
			\arrow{stealth},
			\node[#2] {#1};
		}
	}, \thickness,
	postaction={decorate}
},
latemark/.style 2 args={
	decoration={             
		markings, 
		mark=at position {0.8} with { 
			\arrow{stealth},
			\node[#2] {#1};
		}
	}, \thickness,
	postaction={decorate}
},
zpath/.style={
	decoration={             
		markings, 
		mark=at position {0.5} with { 
			\arrow{stealth},
			\node[#1] {$z^{-1}$};
		}
	}, \thickness,
	postaction={decorate}
},
terminal/.style 2 args={draw,circle,inner sep=2pt,label={#1:#2}},
}


\tikzset{
	invisible/.style={opacity=0},
	visible on/.style={alt={#1{}{invisible}}},
	alt/.code args={<#1>#2#3}{%
		\alt<#1>{\pgfkeysalso{#2}}{\pgfkeysalso{#3}} % \pgfkeysalso doesn't change the path
	},
}

\newcommand\PlotSampledSpectrum[4]{%
	\def\fs{#2}%
	\def\fmax{#3}%
	\def\ros{#4}%
	\input{#1}%
}

\pgfmathdeclarefunction{invgauss}{2}{%
	\pgfmathparse{sqrt(-2*ln(#1))*cos(deg(2*pi*#2))}%
}

\tikzset{
	declare function={
		sinc(\x) = (and(\x!=0, 1) * (sin(deg(pi*\x))/(pi*\x)) +
		(and(\x==0, 1) * 1);
	}
}

\DeclareMathOperator{\E}{\mathbb{E}} % expectation

\newcolumntype{M}[1]{>{\centering\arraybackslash}m{#1}}

\definecolor{blue2}{RGB}{51, 105, 232}  
\definecolor{red2}{RGB}{213, 15, 37}  
\definecolor{green2}{RGB}{0, 153, 37}  
\definecolor{green3}{rgb}{0.1922, 0.6392, 0.3294}% 
\definecolor{yellow2}{RGB}{238, 178, 17} 
\definecolor{gray2}{RGB}{102, 102, 102}
\definecolor{orange2}{RGB}{230, 85, 13}

% Qualitative pallete set1 from www.ColorBrewer.org
\definecolor{Qred}{RGB}{228,26,28}
\definecolor{Qblue}{RGB}{55,126,184}
\definecolor{Qgreen}{RGB}{77,175,74}
\definecolor{Qpurple}{RGB}{152,78,163}
\definecolor{Qorange}{RGB}{255,127,0}
\definecolor{Qyellow}{RGB}{255,255,51}
\definecolor{Qbrown}{RGB}{166,86,40}
\definecolor{Qpink}{RGB}{247,129,191}
\definecolor{Qgray}{RGB}{153,153,153}

\newcommand\SimpleSys[4]{%
	\def\xin{#2}%
	\def\Hz{#3}%
	\def\yout{#4}
	\input{#1}%
} % some definitions

%% 
\title[EE 264]{Power Spectrum Density Estimation}
\author{Jose Krause Perin}
\institute{Stanford University}
\date{August 15, 2017}

\begin{document}

\begin{frame}
  \titlepage
\end{frame}

%
\begin{frame}{Announcements}
\begin{itemize}
	\item Homework \# 6 due Thursday by noon.
	\item Last lecture on Thursday will be a review
	\item Try the practice final exam questions
\end{itemize}
\end{frame}

%
\begin{frame}{Last lecture}
\begin{itemize}
	\item Leakage and resolution are important considerations in spectrum analysis
	\item By choosing proper windows we can minimize these issues
	\item Kaiser window is a nearly optimal choice. Must choose correct $\beta$ and window length $L$
	\item $\beta$ controls the ratio between the amplitudes of the main-lobe and the largest side-lobe i.e., $\beta$ controls the amount of leakage.
	\item The larger the main-lobe width, the smaller the resolution
	\item By increasing the window length we reduce the main-lobe width and consequently improve the resolution
	\item Time-dependent Fourier transform or short-time Fourier transform allows us to keep track of frequency variation in time
	\item Spectrogram is a commonly used way to display the TDFT
	\item In the spectrogram the TDFT is sampled both in time and in frequency
	\item The window length determines the resolution of the spectrogram
\end{itemize}
\end{frame}

%
\begin{frame}{Today's lecture}
	How to estimate the power spectrum density (PSD) of stationary signals.
	\tableofcontents
\end{frame}

\begin{frame}{Stationarity random signals}
	\begin{itemize}
		\item Stationarity refers to time-invariance of some or all statistics of a random process
		\item In practice, we typically work with wide-sense stationary random processes (WSS)
		\item The mean of a WSS process is constant, while the autocorrelation function depends only on the time deference $m$, and not on the absolute time $n$.
		\begin{align}
			\E(x[n]) &=\mu \tag{constant mean} \\
			\E(x[m+n]x^*[n]) &= \phi_{xx}[m] \tag{autocorrelation function depends only on time difference $m$}
		\end{align}
		\item The power spectrum density (PSD) of a WSS process is the Fourier transform of the autocorrelation function:
		\begin{equation*}
			\Phi_{xx}(e^{j\omega}) = \mathcal{F}\{\phi_{xx}[m]\}
		\end{equation*}
		
		The PSD is purely real, even symmetric, and non-negative. 
		
		\begin{equation}
			\E(|x[n]|^2) = \phi_{xx}[0] = \frac{1}{2\pi}\int_{-\pi}^{\pi} \Phi_{xx}(e^{j\omega}) d\omega \tag{average power}
		\end{equation}
	\end{itemize}
\end{frame}

\section{Periodogram}
\section{Welch method}
\section{Blackman-Tukey method}

%
\begin{frame}{Summary}
	\begin{itemize}
		\item a
	\end{itemize}
\end{frame}

\end{document}
