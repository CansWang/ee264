\documentclass[10pt]{article}
\usepackage{geometry}
 \geometry{a4paper}
\usepackage{fancyhdr} % Required for custom headers
\usepackage{lastpage} % Required to determine the last page for the footer
\usepackage{extramarks} % Required for headers and footers
\usepackage[usenames,dvipsnames]{color} % Required for custom colors
\usepackage{graphicx} % Required to insert images
\usepackage{listings} % Required for insertion of code
\usepackage{courier} % Required for the courier font
\usepackage{lipsum} % Used for inserting dummy 'Lorem ipsum' text into the template
\usepackage{caption}
\usepackage{subcaption}
\usepackage{amsmath}
\usepackage{amsmath}
\usepackage{amssymb}
\usepackage{epstopdf}
\usepackage{placeins}
\usepackage{color} 
\usepackage{fancyvrb} 
\usepackage{setspace}
\usepackage[numbered]{bookmark}
\usepackage{pdfpages}
\usepackage{enumitem}
\usepackage{tikz}
\usepackage{pgfplots}
\DeclareGraphicsExtensions{.pdf,.png,.jpg}
\graphicspath{{../figs/}}

\usepgfplotslibrary{fillbetween}
\usetikzlibrary{positioning}

\usetikzlibrary{pgfplots.groupplots}
\usetikzlibrary{plotmarks}
\usetikzlibrary{calc}

\usepgfplotslibrary{groupplots}

\pgfplotsset{compat=newest} 

\DeclareMathOperator{\E}{\mathbb{E}}

\setlength{\parindent}{0pt}

\singlespacing

% Margins
\topmargin=-0.45in
\evensidemargin=0in
\oddsidemargin=0in
\textwidth=6.5in
\textheight=9.0in
\headsep=0.25in

\begin{document}
\doublespacing
\begin{center}
	\textbf{\large Stanford University}
	
	\textbf{\large EE 264: Digital Signal Processing}
	
	\textbf{\large Summer, 2018}
	
	\vspace{2mm}
	
	\textbf{\large Syllabus}
\end{center}
\vspace{-5mm}
\rule{\textwidth}{0.5pt}
\singlespacing

\subsection*{Teaching staff} 

\subsubsection*{Instructor: Jose Krause Perin} 
Office hours: Mondays 8:30 -- 9:30 AM, in \href{https://campus-map.stanford.edu/?id=04-040&lat=37.42879024&lng=-122.1740029&zoom=17&srch=Spilker}{Spilker 145}, or by appointment. \\
SCPD students may join the meeting remotely using Webex with the \href{https://stanford.webex.com/stanford-en/j.php?MTID=m07046ca0da0ed2a46b9a203fa6707fb4}{URL} or by calling 650-227-4712. The meeting number is 929 791 103 and the password is ``EE264'' (without quotation marks). In case you are in a different time zone, please feel free to schedule an appointment at a different time. \\
e-mail: jkperin@stanford.edu

\subsubsection*{Teaching assistant: Anchit Suri} 
Office hours: Thursdays 2:00 -- 3:00 PM, in  \href{https://campus-map.stanford.edu/?id=04-030&lat=37.42924401&lng=-122.17383732&zoom=17&srch=packard}{Parkard 106}, or by appointment. \\
SCPD students may join office hours remotely using Webex: \href{https://stanford.webex.com/meet/anchit}{https://stanford.webex.com/meet/anchit}. \\
e-mail: anchit@stanford.edu

\subsection*{Class meetings} 
\begin{itemize}
	\item Mondays and Wednesdays, 9:30 -- 11:20 AM at \href{https://campus-map.stanford.edu/?srch=Gates+Computer+Science}{Gates B3}.
	\item There will be a midterm review session on Friday, July 20th, from 11:30 AM-- 12:20 PM in \href{https://campus-map.stanford.edu/?srch=Gates+Computer+Science}{Gates B3}.
	\item The review session for the final exam will be on the last day of classes.
\end{itemize}

\subsection*{Tentative schedule and class topics}

\begin{tabular}{c|c|l|c|c}
	\hline
	Date & Lecture	& Lecture Topic	& Reading\footnotemark & Assigned \\
	\hline
	25-Jun	& 1 &	Introduction and review			& Chaps. 2, 3	& \\
	27-Jun	& 2 &	Discrete-time random signals	& 2.10, App. A & HW1\\
	29-Jun 	& 3 & 	Properties of LTI systems 		& 5.1--5.7 	&  \\
	\hline
	2-Jul	& 	& 	\textbf{No lecture (make-up lecture on Jun 29)}	&  & \\
	4-Jul	&  & 	Independence Day, no classes  	&		& HW2 \\
	\hline
	9-Jul	& 4 & 	Sampling and reconstruction  	& 4.1--4.4 	& \\
	11-Jul	& 5	& 	Changing the sampling rate by digital filtering	& 4.6--4.9 		& HW3 \\
	\hline
	16-Jul	& 6	& 	Digital filter structures				& 6.1--6.6 & \\
	18-Jul	& 7	& 	Quantization and finite-precision arithmetic &	6.7--6.10	& \\
	20-Jul	& 	& 	Midterm review session (11:30 AM--11:20 PM in \href{https://campus-map.stanford.edu/?srch=Gates+Computer+Science}{Gates B3})  & & \\
	\hline
	23-Jul	& 	& 	In-class midterm exam (Lecs. 1--7, HWs 1--3) & & \\
	25-Jul	& 8	& 	Filter design & Chap. 7	& HW4 \\
	\hline
	30-Jul	& 9 & 	Adaptive signal processing & & \\
	1-Aug	& 10 & 	The discrete Fourier transform (DFT) & 8.1--8.7, 9.1--9.3	& HW5 \\
	\hline
	6-Aug	& 11 & 	Time-dependent Fourier transform	& 10.1--10.4 & \\
	8-Aug	& 12 & 	Power spectrum density estimation & 10.5, 10.6 & HW6 \\
	\hline
	13-Aug	& 13 & 	Parametric signal modeling 	& 11.1--11.6	& \\
	15-Aug 	& 14 & 	Review and conclusions &  & 	\\
	TBD	& 	 & 	24-hour-take-home final exam & \\
	\hline
\end{tabular}

\footnotetext[1]{Reading assignments in ``Discrete-time signal processing'', Oppenheim and Schafer, 3rd edition}.

\subsection*{Resources}

\subsubsection*{Textbook}
\begin{itemize}
	\item ``Discrete-time signal processing'', Oppenheim and Schafer, 3rd edition, 2010.
	\item Available on 4-hour reserve at the \href{https://campus-map.stanford.edu/?id=04-080&lat=37.42787956&lng=-122.17429865&zoom=17&srch=engineeri}{Engineering Library (Terman)}.
	\item Lecture notes will cover all the material, but further reading of the textbook is encouraged.
\end{itemize}

\subsubsection*{Class website}

The class website \href{https://canvas.stanford.edu}{https://canvas.stanford.edu} contains

\begin{itemize}
	\item Lecture notes, homework assignments and solutions, Matlab code, discussion.
	\item Homework submission will be online on Canvas.
\end{itemize}

\subsection*{Grading}
Homework: $30\%$ \\
Midterm: $30\%$ \\
Final: $40\%$ 

\subsection*{Homework policy}

There will be 6 homework assignments.
\begin{itemize}
	\item Most problem sets will be assigned on Friday and will be due the following Friday by 11:59 PM.
	\item Assignments should be submitted online on Canvas. Submit a single .pdf file with your solutions. 
	\item Homework solutions will typically be available shortly after the homework due date. Homework grades will typically be available within a week of the due date.
	\item Discussion among students is encouraged, but individual solutions must be submitted
	\item In general, late homeworks will not be accepted. Under extenuating circumstances, extensions must be approved by the instructor or TA, and arranged prior to the deadline.
\end{itemize}

\subsection*{Exams policy}
\subsubsection*{Midterm}
\begin{itemize}
	\item In-class midterm on July 23. 
	\item There will be a midterm review session on Friday, July 20th, from 11:30 AM-- 12:20 PM in \href{https://campus-map.stanford.edu/?srch=Gates+Computer+Science}{Gates B3}.
	\item Midterm covers lectures 1 to 7 (homeworks 1 to 3).
	\item Midterm is open book and open notes.
\end{itemize}

\subsubsection*{Final}
\begin{itemize}
	\item The final exam will be a 24-hour-take-home exam. Dates TBD.
	\item The final exam should take just a few hours, and it will involve implementation and simulation of some digital signal processing algorithms.
	\item The review session for the final will be on the last lecture.
\end{itemize}

\end{document}
